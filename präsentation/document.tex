%% LaTeX Beamer presentation template (requires beamer package)
%% see http://latex-beamer.sourceforge.net/
%% idea contributed by H. Turgut Uyar
%% template based on a template by Till Tantau
%% this template is still evolving - it might differ in future releases!

\documentclass[14pt]{beamer}

\mode<presentation>
{
\usetheme{Berlin}

\setbeamercovered{transparent}
}

\usepackage[ngerman]{babel}
\usepackage[latin1]{inputenc}


% font definitions, try \usepackage{ae} instead of the following
% three lines if you don't like this look
%%%\usepackage{mathptmx}
%%%\usepackage[scaled=.90]{helvet}
%%%\usepackage{courier}

\ifx\pdftexversion\undefined
\usepackage[dvips]{graphicx}
\else
\usepackage{graphicx}
\DeclareGraphicsRule{*}{mps}{*}{}
\fi

\usepackage{graphics}
%\usepackage{makeidx}
%\usepackage{mathpazo}
%\usepackage{multicol}
%\usepackage{srcltx}
\usepackage{color}



\title{ATTAC | Was wollen die Globalisierungsgegner?}

%\subtitle{}

% - Use the \inst{?} command only if the authors have different
%   affiliation.
%\author{F.~Author\inst{1} \and S.~Another\inst{2}}
\author{Bj\"{o}rn Schr\"{o}der \and Oluf Lorenzen\inst{}}

% - Use the \inst command only if there are several affiliations.
% - Keep it simple, no one is interested in your street address.
\institute
{
\inst \normalfont
Multi Media\\
\normalfont
Berufsbildende Schulen\\
}

\date{Politik WPK, Hr. Thomae}


% This is only inserted into the PDF information catalog. Can be left
% out.
%%%\subject{Talks}



% If you have a file called "university-logo-filename.xxx", where xxx
% is a graphic format that can be processed by latex or pdflatex,
% resp., then you can add a logo as follows:

\pgfdeclareimage[height=0.5cm]{mmbbslogo}{mmbbslogo}
 \logo{\pgfuseimage{mmbbslogo}}



% Delete this, if you do not want the table of contents to pop up at
% the beginning of each subsection:
%%\AtBeginSubsection[]
%{
%\begin{frame}<beamer>
%\frametitle{Vorgehensweise}
%\tableofcontents[currentsection,currentsubsection]
%\end{frame}
%}

% If you wish to uncover everything in a step-wise fashion, uncomment
% the following command:

%\beamerdefaultoverlayspecification{<+->}

\begin{document}

\begin{frame}
\titlepage
\end{frame}

\begin{frame}
\frametitle{Vorgehensweise}
\tableofcontents
% You might wish to add the option [pausesections]
\end{frame}


\section{Entstehung von ATTAC}

\subsection[Gr\"{u}ndung]{Gr\"{u}ndung}

%\begin{frame}
%\frametitle{}
%\framesubtitle{Subtitles are optional}
%
%\begin{itemize}
%  \item
%  \item
%\end{itemize}
%\end{frame}

\begin{frame}
%\frametitle<presentation>{Gr\"{u}ndung}

\begin{itemize}
  \item \textbf{a}ssociation pour une \textbf{t}axation des \textbf{t}ransactions financi\`{e}res pour l'\textbf{a}ide aux \textbf{c}itoyens\\
  \textit{Vereinigung f\"{u}r eine Besteuerung von Finanztransaktionen zum Nutzen der B\"{u}rger}
  \item Tobin-Steuer
  \item diverse andere Ziele u.a. fairer Welthandel, gleichbehandlung der Geschlechter
\end{itemize}
\end{frame}

\begin{frame}
 \begin{itemize}
  \item jedes halbe Jahr Hauptversammlung, sog. "`Ratschlag"'
  \item Handelm im Konsens, keine Mehrheitsentscheidung
  \item Streben nach "`gutem Kapitalismus"', keine generelle Ablehnung
 \end{itemize}

\end{frame}


\begin{frame}
\frametitle{ATTAC in Deutschland}
\begin{itemize}
  \item Br\"{u}ckenschlag gelang Anfang 2000 unter dem Aspekt der Zusammenarbeit
\end{itemize}
\end{frame}

%\subsection{erste Aktionen}
%\begin{frame}
%\begin{itemize}
%  \item 1900 sdasd  asdqqrsdasd  asdqqrsdasd  asdqqr
%  \item 1902 asdasd  asdqqr
%\end{itemize}
%\end{frame}
\subsection{Aktionen in Deutschland}
\begin{frame}
%\frametitle{erste Aktionen in Deutschland}
\begin{itemize}
  \item Anfang 2002 |\\
      Kampagne gegen Privatisierung d. Gesundheitswesens, Ausweitung Mitte 2002
 			 % zweitens dann mit Aktionstagen in 20 Staedten
  \item Anfang 2002 bis Ende 2002 |\\
      Ver\"{o}ffentlichung der GATS-Forderungen, Informations-Kampagne
 			 % privatisierung d. Bildungssektors, kommunale Dienstleisungen
  \item Anfang 2003 |\\
      diverse Aktionen zum Krieg im Irak 
\end{itemize}
\end{frame}

\begin{frame}
%\frametitle{erste Aktionen in Deutschland}
\begin{itemize}
  \item Ende 2003 | Beginn von Kampagnen gegen Softwarepatente
  \item Ende 2003 | Aktionen gegen Harz 4 und Sozialabbau

\end{itemize}
\end{frame}


\section{H\"{o}hepunkte}

\begin{frame}
\frametitle{H�hepunkte}
  \begin{itemize}
    \item G8-Gipfel in Genua, durch gro�e Teilnahme der Medien Mitgliederwachstum
    \item Teilnahme am Weltsozialforen, z.B. in Davos
    \item Demonstrationen gegen \"{U}bernahme Mannesmann durch Vodafone
  \end{itemize}
\end{frame}

\section{Zuk\"{u}nftige Laufbahn}
\begin{frame}
\frametitle{Zuk\"{u}nftige Laufbahn}
\begin{itemize}
  \item \textit{european-summer-university.eu} Attac-Sommeruniversit�t
  \item Erreichen neuer Ziele
  \begin{itemize}
    \item Stopp der Bahnprivatisierung
    \item Umverteilung von Reicht�mern
    \item Ver�nderung des Gedankenansatzes die 3. Welt auszubeuten
  \end{itemize}
\end{itemize}
\end{frame}


\section[Resum�e]{Zusammenfassung: was also sind die Ziele?}
\subsection[]{}
\begin{frame}
\frametitle{Resum\'{e}e}
%\begin{itemize}
%  \item 
%\end{itemize}
\end{frame}

%\subsection[]{Wirtschaft}
%\begin{frame}
%\frametitle{Wirtschaft}
%\begin{itemize}
%  \item 
%\end{itemize}
%\end{frame}

%\subsection[]{kombiniert?}
%\begin{frame}
%\frametitle{kombiniert?}
%\begin{itemize}
%  \item 
%\end{itemize}
%\end{frame}


\section[]{}

\begin{frame}
%\frametitle<presentation>{Quellen}
\frametitle{Quellen}

\begin{itemize}
  %\item The \alert{first main message} of your talk in one or two lines.
  \item www.attac.\{de,org,ch\}
  \item www.kommunikationssystem.de
  \item www.monde-diplomatique.de
\end{itemize}
  \vskip0pt plus.5fill
  \ldots erstellt mit \LaTeX
\end{frame}

\section[]{}
\begin{frame}
\Large Vielen Dank f\"{u}r eure Aufmerksamkeit \& Kritik!
\end{frame}


% You can create overlays
%\begin{itemize}
%  \item using the \texttt{pause} command:
%  \begin{itemize}
%    \item First item.
%    \pause
%    \item Second item.
%  \end{itemize}
%  \item using overlay specifications:
% \begin{itemize}
%    \item<3-> First item.
%    \item<4-> Second item.
%  \end{itemize}
%  \item using the general \texttt{uncover} command:
%  \begin{itemize}
%    \uncover<5->{\item First item.}
%    \uncover<6->{\item Second item.}
%  \end{itemize}
%\end{itemize}


% The following outlook is optional.
%\vskip0pt plus.5fill
%\begin{itemize}
%  \item Outlook
%  \begin{itemize}
%    \item Something you haven't solved.
%    \item Something else you haven't solved.
%  \end{itemize}
%\end{itemize}


\end{document}
